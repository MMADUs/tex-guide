% define the document type
% like article, report, book, etc.
\documentclass{article}

% use a math formatting package
\usepackage{amsmath}

% title of the document
\title{My First LaTeX Document}
% author of the document
\author{Your Name}
% set the date on document (today for ex)
\date{\today}

% start of the document content
% any content is written inside the document section
\begin{document}

% the \maketitle is mandatory
% used to generate the header metadata
% such as the title, author, and date.
\maketitle

% the \section used to make a new content section
% creates a new given section title
% the section number will be auto-incremented
% ex:
% 1. section one
% 2. section two
% any content inside the \section
% will be the part of the section content
\section{Introduction}

This is my first document in \LaTeX{}. I can write text here and format it easily.

% similiar to \section but this generates the sub part
% the number format ex:
% 1.1 sub one
% 1.2 sub two
\subsection{Why Use LaTeX?}
LaTeX is great for:

% the itemsize is used to create bullet points
% such of making a list of items
\begin{itemize}
    \item Writing academic papers
    \item Creating mathematical equations
    \item Formatting text professionally
\end{itemize}

\section{Mathematics in LaTeX} % Another section
LaTeX makes it easy to write equations. For example:

\[
E = mc^2
\]

Here's an inline equation: \( a^2 + b^2 = c^2 \).

\section*{Matrix Example}

Here is a simple \( 3 \times 3 \) matrix:

\[
A =
\begin{pmatrix} % bmatrix creates brackets around the matrix
1 & 2 & 3 \\
4 & 5 & 6 \\
7 & 8 & 9
\end{pmatrix}
\]

This is an inline equation: \( a^2 + b^2 = c^2 \).
	

You can also create a matrix without brackets:

\[
B =
\begin{matrix} % Simple matrix with no brackets
1 & 2 & 3 \\
4 & 5 & 6 \\
7 & 8 & 9
\end{matrix}
\]

\end{document} % End of the document